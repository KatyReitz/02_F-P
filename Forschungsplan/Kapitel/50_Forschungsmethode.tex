\section{Forschungsmethode}
\label{sec:Forschungsmethode}
\begin{itemize}
  \item Allgemein
  \begin{itemize}
    \item Beschreibung der Forschungsmethode: qualitativ, quantitativ, konstruktiv, verhaltenswissenschaftlich
    \item Begründung des Vorgehens
  \end{itemize}
  \item Verhaltenswissenschaftliche Ansätze:
  \begin{itemize}
    \item Vorgehen, z.B. explorative Studie, Testen von Hypothesen
    \item Art der Untersuchung, z.B. initiale Klärung, Beschreibung von Kausalzusammenhängen
    \item Zeithorizont: einmalig, longitudinal 
    \item Aufbau der Studie, Einfluss des Forschenden, z.B. Feldexperiment, Kontrolle der Rahmenbedingungen
    \item Untersuchungseinheit und Stichprobenauswahl, z.B. Individualpersonen, zufällige Auswahl
    \item Messgrößen: Definitionen, Metrik, Maßeinheiten
    \item Datenerfassung, z.B. Interviews, Beobachtung, Fragebogen; Beispielhafte Beschreibung der erwarteten Daten
    \item Datenanalyse: Anwendung qualitativer oder quantitativer Methoden; Beschreibung des Vorgehens
  \end{itemize}
  \item Konstruktivistische Ansätze:
  \begin{itemize}  
    \item Beschreibung der Ergebnisartefakte (Umfang und Detaillierungsgrad); Darstellung konkreter hypothetischer Beispiele
    \item Beschreibung der Methode zur Erstellung der Ergebnisartefakte; Skizze der wesentlichen Schritte
    \item Beschreibung der Methode zur Evaluation der Ergebnisse, z.B. mathematische Analyse, Beobachtung, Laborexperiment, Test, Fallstudie; Skizze der wesentlichen Schritte 
  \end{itemize}
\end{itemize}